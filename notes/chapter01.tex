\chapter{向量}

\section{向量的几何表示}

\subsection{行图像}

行图像是线性方程组所构成的图像,在二元线性方程组中,每个方程组代表一条直线,求接就是求二线的交点。

\subsection{列图像}

列图像是列向量,在图像中表现为向量的运算。

\subsection{矩阵图像}

矩阵图像:$\MA x=\Sb$

\section{向量长度和点积}

\subsection{点积}

\kt{dot product} or \kt{inner product}

$$\Sv \cdot \Sw = \sum_{i=1}^{n}\Sv_i \cdot \Sw_i$$

\subsection{长度}

\kt{length}就是向量的范数:

$$\norm{\Sv} = \sqrt{\Sv \cdot \Sv}$$

长度为1的向量是单位向量 \kt{unit vector}。

\section{矩阵}

\subsection{矩阵乘法}

$$\MA \MB = \MC$$

$$c_{ij} = \sum_{k=1}^{n}a_{ik}b_{kj}$$

\kw{分块}

\begin{equation}
        \begin{bmatrix}
            A_{1} & A_{2} \\
            A_{3} & A_{4} \\
        \end{bmatrix} 
        \begin{bmatrix}
            B_{1} & B_{2} \\
            B_{3} & B_{4} \\
        \end{bmatrix} = 
        \begin{bmatrix}
            C_{1} & C_{2} \\
            C_{3} & C_{4} \\
        \end{bmatrix} 
\end{equation}

$$C_1 = A_1B_1 + A_2B_3$$

\subsection{逆矩阵}

$$\MA \MA^{-1} = \MI$$